\documentclass[
  xcolor={svgnames},
  hyperref={colorlinks,citecolor=DeepPink4,linkcolor=DarkRed,urlcolor=DarkBlue}
  ]{beamer}

% define using customized theme.
\usetheme{pas}

% define using packages
\usepackage[utf8]{inputenc}
\usepackage[T1]{fontenc}
\usepackage {minted}

% the general information.
\title[] % (optional, only for long titles)
{Citation Intent Classification}
\subtitle{Identifying the Intent of a Citation in scientific papers}

\author[tmip, hieutt] % (optional, for multiple authors)
{Isaac Riley and Pavan Mandava}
\institute[Universities Here and There] % (optional)
{
  \inst{1}%
  Computational Linguistics, M.Sc.\\
  \and
  \inst{2}%
  Computational Linguistics, M.Sc.\\
}
\date[] % (optional)
{May 20, 2020}
\subject{Computational Linguistics}



% begin presentation content
\begin{document}

%%%% Slide : 1 -- INTRO
\begin{frame}
\titlepage
\end{frame}


%%%% TASK DESCRIPTION ----- Slide 2
\begin{frame}
\frametitle{Task Description}

\begin{itemize}

\item Identifying intent of a citation in scientific papers
\bigskip
\item Three Intent categories/classes from the data set
	\begin{enumerate}
		\item background (background information)
		\item method (use of methods/tools)
		\item result (comparing results)
	\end{enumerate}
\bigskip
\item {\bf Classification Task }
	\begin{itemize}
		\item Assign a discrete class (intent) for each data point
	\end{itemize}
\end{itemize}

\end{frame}

%%%% DATA SET ---- Slide 3
\begin{frame}
\frametitle{Data set}

\begin{itemize}
\item Training Data: 8.2K+ data points
	\begin{enumerate}
		\item background - 4.8K
		\item method - 2.3K
		\item result - 1.1K
	\end{enumerate}
\item Testing Data: 1.8K data points
	\begin{enumerate}
		\item background - 1K
		\item method - 0.6K
		\item result - 0.2K
	\end{enumerate}
\end{itemize}

\begin{table}
	\begin{tabular}{| l | c | c} \hline
		 4 lead to a decrease in SC absorption in mice & \\
		 (Deng et al., 2010; Deng et al., 2012). & background \\
		  \hline
		 We used an active contour algorithm [10] to segment & \\
		 organs from 340 coronal slices over the two patients. & method \\
		  \hline
		 Similar results were found by Sideris et al. (1999) in & \\
		 Greece and Mohebali et al. (2005) in Iran. & result \\
		  \hline
	 \end{tabular}

\caption{Sample Dataset}
\end{table}

\end{frame}


%%%% Approach/Architectures ---- Slide 4
\begin{frame}[fragile]
\frametitle{Approach \& Architecture}
\framesubtitle{Classifier Implementation}

Base Classifier: {\bf {\color{red} Perceptron}}
\begin{itemize}
	\item Linear Classifier
	\item Binary Classifier
\end{itemize}

\begin{minted}[autogobble, breaklines,breakanywhere, fontfamily=helvetica, fontsize=\small]{python}
class Perceptron:
   def __init__(self, label: str, weights: dict, theta_bias: float)
   def score(self, features: list)
   def update_weights(self, features: list, learning_rate: float, penalize: bool, reward: bool)

class MultiClassPerceptron:
   def __init__(self, epochs: int,learning_rate: float,random_state: int)
   def fit(self, X_train: list, labels: list)
   def predict(self, X_test: list)

\end{minted}
\bigskip
- {\bf Parameters} and {\bf Hyperparameters}

\end{frame}



%%%% Approach/Architectures ---- Slide 5
\begin{frame}[fragile]
\frametitle{Approach \& Architecture}
\framesubtitle{Feature Representation}

Lexicons and Regular Expressions ($\approx$ 30 Features)
\bigskip

\begin{itemize}
	\item LEXICONS
		\begin{minted}[autogobble, breaklines,breakanywhere, fontfamily=helvetica, fontsize=\small]{python}
		ALL_LEXICONS = {
		    'INCREASE': ['increase', 'grow', 'intensify', 'build up', 'explode'],
		    'USE': ['use', 'using', 'apply', 'applied', 'employ', 'make use'],
		    .....
		}
		\end{minted}
	\bigskip
	\item REGEX
		\begin{itemize}
		    \item $ACRONYM$
		    \item $CONTAINS\_URL$
		    \item $ENDS\_WITH\_ETHYL$
		\end{itemize}
\end{itemize}

\end{frame}

%%%% Evaluation ---- Slide 6
\begin{frame}[fragile]
\frametitle{Evaluation of the Classifier}
\framesubtitle{F1 Score}

\bigskip

\begin{itemize}
	\item F1 Score
		\begin{itemize}
			\item weighted average of Precision and Recall
		\end{itemize}
		\bigskip
		\begin{minted}[autogobble, breaklines,breakanywhere, fontfamily=helvetica]{python}
		def f1_score(y_true, y_pred, labels, average)
		\end{minted}
		\bigskip
	\item Averaging
		\begin{itemize}
			\item MACRO
			\item MICRO
			\item None
		\end{itemize}
	\bigskip
	\item Why {\color{red} MACRO} and {\color{red}MICRO} ?
	
\end{itemize}

\end{frame}

%%%% Results ---- Slide 7
\begin{frame}
\frametitle{Model Performance}
\framesubtitle{Results}

\begin{table}
\def\arraystretch{1.5}
	{\setlength{\tabcolsep}{2em}
	\begin{tabular}{| l | c | c |} \hline
		 {\bf Averaging} & {\bf Score} \\
		 \hline \hline
		 MICRO & 0.64 \\
		  \hline
		 MACRO & 0.57 \\
		  \hline
		 background & 0.72 \\
		 method & 0.54 \\
		 result & 0.46 \\
		  \hline
	 \end{tabular}}

\caption{F1-Score Results}
\end{table}

\end{frame}


%%%% Next Steps ---- Slide 8
\begin{frame}
\frametitle{Next Steps}
	\begin{itemize}
		\item Better Feature Representation - Word Embeddings
		\begin{itemize}
			\item word2vec
			\item BERT
			\item ELMo
			\item \dots{}
		\end{itemize}
		
		\item Better Classifier (Non-Linear / Neural Networks)
		\begin{itemize}
			\item BiRNNs
			\item BiLSTMs
			\item CNNs
			\item \dots{}
		\end{itemize}
		
		\item Interaction with other groups
	\end{itemize}
\end{frame}

%%%% THANK YOU -- Slide 9
\begin{frame}
\usebeamerfont{frametitle}\usebeamercolor[fg]{frametitle}
  \centering \Large
  	Thanks for listening
\end{frame}

\end{document}


