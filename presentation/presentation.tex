\documentclass[
  xcolor={svgnames},
  hyperref={colorlinks,citecolor=DeepPink4,linkcolor=DarkRed,urlcolor=DarkBlue}
  ]{beamer}

% define using customized theme.
\usetheme{pas}

% define using packages
\usepackage[utf8]{inputenc}
\usepackage[T1]{fontenc}
\usepackage {minted}

% the general information.
\title[] % (optional, only for long titles)
{Citation Intent Classification}
\subtitle{Identifying the Intent of a Citation in scientific papers}

\author[tmip, hieutt] % (optional, for multiple authors)
{Isaac Riley and Pavan Mandava}
\institute[Universities Here and There] % (optional)
{
  \inst{1}%
  Computational Linguistics, M.Sc.\\
  \and
  \inst{2}%
  Computational Linguistics, M.Sc.\\
}
\date[] % (optional)
{May 20, 2020}
\subject{Computational Linguistics}



% begin presentation content
\begin{document}

%%%% Slide : 1 -- INTRO
\begin{frame}
\titlepage
\end{frame}


%%%% TASK DESCRIPTION ----- Slide 2
\begin{frame}
\frametitle{Task Description}

\begin{itemize}

\item Identifying intent of a citation in scientific papers
\item Three Intent categories/classes from the data set
	\begin{enumerate}
		\item background (background information)
		\item method (use of methods/tools)
		\item result (comparing results)
	\end{enumerate}
\item Classification Task 
	\begin{itemize}
		\item Assign a discrete class (intent) for each data point
	\end{itemize}
\end{itemize}

\end{frame}

%%%% DATA SET ---- Slide 3
\begin{frame}
\frametitle{Data set}

\begin{itemize}

\item Training Data: 8.2K+ data points
	\begin{enumerate}
		\item background - 4.8K
		\item method - 2.3K
		\item result - 1.1K
	\end{enumerate}
\item Testing Data: 1.8K data points
	\begin{enumerate}
		\item background - 1K
		\item method - 0.6K
		\item result - 0.2K
	\end{enumerate}
\end{itemize}

\end{frame}


%%%% Approach/Architectures ---- Slide 4
\begin{frame}[fragile]
\frametitle{Approach \& Architecture}
\framesubtitle{Classifier Implementation}

Base Classifier: {\bf {\color{red} Perceptron}}
\begin{itemize}
	\item Linear Classifier
	\item Binary Classifier
\end{itemize}
\bigskip

\begin{minted}[autogobble, breaklines,breakanywhere, fontfamily=helvetica, fontsize=\small]{python}
class Perceptron:
   def __init__(self, label: str, weights: dict, theta_bias: float)
   def score(self, features: list)
   def update_weights(self, features: list, learning_rate: float, penalize: bool, reward: bool)

class MultiClassPerceptron:
   def __init__(self, epochs: int,learning_rate: float,random_state: int)
   def fit(self, X_train: list, labels: list)
   def predict(self, X_test: list)

\end{minted}

\end{frame}



%%%% Approach/Architectures ---- Slide 5
\begin{frame}[fragile]
\frametitle{Approach \& Architecture}
\framesubtitle{Feature Representation}

Lexicons and Regular Expressions

\begin{itemize}
	\item LEXICONS
	\item REGEX
\end{itemize}



\end{frame}



\begin{frame}
\frametitle{There Is No Largest Prime Number}
\framesubtitle{The proof uses \textit{reductio ad absurdum}.}
\begin{theorem}
There is no largest prime number. \end{theorem}
\begin{enumerate}
\item<1-| alert@1> Suppose $p$ were the largest prime number.
\item<2-> Let $q$ be the product of the first $p$ numbers.
\item<3-> Then $q+1$ is not divisible by any of them.
\item<1-> But $q + 1$ is greater than $1$, thus divisible by some prime
number not in the first $p$ numbers.
\end{enumerate}
\end{frame}

\begin{frame}{A longer title}
\begin{itemize}
\item one
\item two
\end{itemize}
\end{frame}


\begin{frame}[allowframebreaks]
   \frametitle{References}
   \bibliographystyle{plain}
   \bibliography{lib}
\end{frame}

\end{document}
